\input includes/\format

\pdfinfo{
   /Title (Encyclopedia of Life & Death - Intermediate problems)
   /Creator (TeX)
   /Author (Vit Brunner)
   /CreationDate (20180627233500)
}

~
\title
\parindent=0em
\baselineskip=1em

\vskip 3cm
\leftline{Cho Chikun's}
\medskip
\leftline{Encyclopedia of}
\medskip
\leftline{Life \& Death}
\vskip 2cm
\leftline{\header Part 2 --- Intermediate}
\vfill\break

\input includes/\format-motto

\parindent=1em
\bigskip
\noindent{\header Preface}
\medskip
\noindent This is a collection of almost three thousand problems from Encyclopedia of Life and Death by Cho Chikun. The problems come without solutions for two reasons: first, one can learn more by reading out all the paths and solving the problems oneself; second, the solutions are copyrighted. It is always {\bf black to move}, diagrams are shown without any distracting text around.

In the second part, you can find about nine hundred problems for strong kyu players, but dan players will also benefit from going through the book --- they should find the solution for each problem in less than a minute and are likely to spend about twelve hours solving all the problems.

I wish you enjoyment and improvement in the wonderful game of go, weiqi, baduk, or whatever you like to call it.

\rightline{\it\me}
\rightline{\it November 2004}

\bigskip
\noindent{\header Anniversary edition}
\medskip
\noindent It's been 13 and a half years. This booklet has been downloaded over thirty thousand times. I never imagined the reach it would have. The anniversary edition comes with a better layout and wording. And --- glad you asked --- some of the problems are still unsolvable!

\rightline{\it\me}
\rightline{\it June 2018}
\vfill\break

\input includes/\format-gotc
\input problems/cho-2

\bye
