\input includes/\format

\pdfinfo{
   /Title (Encyclopedia of Life & Death - Elementary problems)
   /Creator (TeX)
   /Author (Vit Brunner)
   /CreationDate (20180627233500)
}

~
\title
\parindent=0em
\baselineskip=1em

\vskip 3cm
\leftline{Cho Chikun's}
\medskip
\leftline{Encyclopedia of}
\medskip
\leftline{Life \& Death}
\vskip 2cm
\leftline{\header Part 1 --- Elementary}
\vfill\break

\input includes/\format-motto

\parindent=1em
\bigskip
\noindent{\header Preface}
\medskip
\noindent This is a collection of almost three thousand problems from Encyclopedia of Life and Death by Cho Chikun. The problems come without solutions for two reasons: first, one can learn more by reading out all the paths and solving the problems oneself; second, the solutions are copyrighted. It is always {\bf black to move}, diagrams are shown without any distracting text around.

In this first part, you'll find about nine hundred problems mostly for beginners, although even dan players can benefit from going through these problems quickly --- a dan player should get the solution in a few seconds and is likely to spend about an hour reading it through.

I wish you enjoyment and improvement in the wonderful game of go, weiqi, baduk, or whatever you like to call it.

\rightline{\it\me}
\rightline{\it November 2004}

\bigskip
\noindent{\header Anniversary edition}
\medskip
\noindent It's been 13 and a half years. This booklet has been downloaded over a hundred thousand times. I never imagined the reach it would have. The anniversary edition comes with a better layout and wording. And --- glad you asked --- some of the problems are still unsolvable!

\rightline{\it\me}
\rightline{\it June 2018}
\vfill\break

\input includes/\format-gotc
\input problems/cho-1

\bye
