\input includes/\format

\pdfinfo{
   /Title (Encyclopedia of Life & Death - Elementary problems)
   /Creator (TeX)
   /Author (Vit Brunner)
   /CreationDate (20180627233500)
}

~
\title
\parindent=0em
\baselineskip=1em

\vskip 3cm
\leftline{Cho Chikun's}
\medskip
\leftline{Encyclopedia of}
\medskip
\leftline{Life \& Death}
\vskip 1em
\leftline{\subtitle Part 1 --- Elementary}
\vfill\break

\input includes/\format-motto

\parindent=1em
\bigskip
\noindent{\header Preface}
\medskip
\noindent This is a collection of almost three thousand problems from Encyclopedia of Life and Death by Cho Chikun. The problems come without solutions for two reasons: first, one can learn more by reading out all the paths and solving the problems oneself; second, the solutions are copyrighted. All the problems are {\bf black to move}.

In this first part, you'll find about nine hundred problems for beginners. A dan player should find the solution in a few seconds and will need about an hour solve the whole book.

I wish you enjoyment and improvement in the wonderful game of go, weiqi, baduk, or whatever you like to call it.

\rightline{\it\me}
\rightline{\it November 2004}

\bigskip
\noindent{\header Anniversary edition}
\medskip
\noindent It's been 13 and a half years. This booklet has been downloaded over a hundred thousand times. I never imagined the reach it would have. The anniversary edition comes with a better layout and wording. And --- glad you asked --- some of the problems are still unsolvable!

\rightline{\it\me}
\rightline{\it June 2018}
\vfill\break

\input includes/\format-gotc
\input problems/cho-1

\bye
