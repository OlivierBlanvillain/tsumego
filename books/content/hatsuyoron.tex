\input includes/\format

\pdfinfo{
   /Title (Igo Hatsuyoron)
   /Creator (TeX)
   /Author (Vit Brunner)
   /CreationDate (20180627233500)
}

~
\title
\parindent=0em
\baselineskip=1em

\vskip 5cm
\centerline{\pdfximage width \titleimagewidth {titles/hatsuyoron.pdf}\pdfrefximage\pdflastximage}
\bigskip
\centerline{Igo Hatsuyo-ron}
\vfill\break

\input includes/\format-motto

\parindent=1em
\bigskip
\noindent{\header Preface}
\medskip
\noindent The notoriously difficult Igo Hatsuyo-ron was compiled around 1710 by Inoue Dosetsu Inseki, fourth head of the Inoue house and fifth Meijin Godokoro. It was designed for training of the best students of the Inoue school and was kept secret for a long time.

Igo Hatsuyo-ron consists of 183 mostly insanely difficult problems and is aimed at serious players with deep interest in the game. While solving the problems takes many months, possibly even years, finding the solution is always particularly rewarding.

I wish you enjoyment and improvement in the wonderful game of go, weiqi, baduk, or whatever you like to call it.

\rightline{\it\me}
\rightline{\it November 2006}

\bigskip
\noindent{\header Anniversary edition}
\medskip
\noindent It's been 11 and a half years. This booklet has been downloaded over twenty thousand times. I never imagined the reach it would have. The anniversary edition comes with a better layout and wording. The problems are still crazy hard!

\rightline{\it\me}
\rightline{\it June 2018}
\vfill\break

\input includes/\format-gotc
\input problems/hatsuyoron

\bye
