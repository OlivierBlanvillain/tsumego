\input includes/\format

\pdfinfo{
   /Title (Gokyo Shumyo)
   /Creator (TeX)
   /Author (Vit Brunner)
   /CreationDate (20180627233500)
}

~
\title
\parindent=0em
\baselineskip=1em

\vskip 5cm
\centerline{\pdfximage width \titleimagewidth {titles/gokyoshumyo.pdf}\pdfrefximage\pdflastximage}
\bigskip
\centerline{Gokyo Shumyo}
\vfill\break

\input includes/\format-motto

\bigskip
{\header Preface}
\medskip
Gokyo Shumyo is a classical problem collection published by Hayashi Genbi in 1822. Its name could be translated to English as ``Brilliances from go classics''\negthinspace. The collection consists of 520 problems, which are divided into the following seven sections:
\bigskip

1. Living (103 problems)\dotfill 1

2. Killing (71 problems)\dotfill 10

3. Creating a ko (90 problems)\dotfill 17

4. Capturing races (96 problems)\dotfill 25

5. Oiotoshi (40 problems)\dotfill 34

6. Connecting (74 problems)\dotfill 38

7. Various techniques (46 problems)\dotfill 46

\bigskip

I wish you to enjoy and improve while studying these problems,

\rightline{\it\me}
\rightline{\it December 2006}

\bigskip
\noindent{\header Anniversary edition}
\medskip
\noindent It's been 11 and a half years. This booklet has been downloaded over twenty thousand times. I never imagined the reach it would have. The anniversary edition comes with a better layout and wording.

\rightline{\it\me}
\rightline{\it June 2018}
\vfill\break

\input includes/\format-gotc
\baselineskip=1.1em
\input problems/gokyoshumyo-mirror

\bye
