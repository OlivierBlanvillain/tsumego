\input include-a5

\pdfinfo{
   /Title (Xuan Xuan Qi Jing)
   /Creator (TeX)
   /Author (Vit Brunner)
   /CreationDate (20180627233500)
}

~
\title
\parindent=0em
\baselineskip=1em

\vskip 5cm
\centerline{\pdfximage width 8cm {titles/xxqj.pdf}\pdfrefximage\pdflastximage}
\bigskip
\centerline{Xuanxuan Qijing}
\bigskip
\centerline{(Gengen Gokyo)}
\vfill\break

\input include-motto

\parindent=1em
\bigskip
\noindent{\header Preface}
\medskip
\noindent {\it Gateway to All Marvels.} Xuanxuan Qijing is a classical Chinese collection compiled by Yan Defu and Yan Tianzhang around year 1347. Its perhaps more popular Japanese name is Gengen Gokyo.

There circulate many versions of the collection. This should be one of the more complete versions, containing 347 problems of a fairly difficult level. Unless you are a top amateur player, some of the problems might be difficult for you. However, if you keep solving the problems, you will improve your reading immensely.

Enjoy solving the problems and improving your reading.
\bigskip
\rightline{\it\me}
\rightline{\it December 2006}

\bigskip
\noindent{\header Anniversary edition}
\medskip
\noindent It's been 11 and a half years. This booklet has been downloaded over twenty thousand times. I never imagined the reach it would have. The anniversary edition comes with a better layout and wording. No fixes necessary!
\bigskip
\rightline{\it\me}
\rightline{\it June 2018}
\vfill\break

\input include-a5-gotc
\input problems-xxqj

\bye
